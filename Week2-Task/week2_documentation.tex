% Setting up the document with necessary packages
\documentclass[a4paper,12pt]{article}
\usepackage[utf8]{inputenc}
\usepackage[T1]{fontenc}
\usepackage{geometry}
\geometry{margin=1in}
\usepackage{listings}
\usepackage{xcolor}
\usepackage{titlesec}

% Configuring code listing style
\lstset{
    language=Python,
    basicstyle=\ttfamily\small,
    keywordstyle=\color{blue}\bfseries,
    stringstyle=\color{red},
    commentstyle=\color{green!60!black},
    numbers=left,
    numberstyle=\tiny,
    stepnumber=1,
    numbersep=5pt,
    showspaces=false,
    showstringspaces=false,
    frame=single,
    breaklines=true,
    breakatwhitespace=true,
    tabsize=4
}

% Customizing section headings
\titleformat{\section}{\large\bfseries}{\thesection}{1em}{}
\titleformat{\subsection}{\normalsize\bfseries}{\thesubsection}{1em}{}

% Document title and author
\title{Week 2 Python Programming Tasks Documentation}
\author{}
\date{July 3, 2025}

\begin{document}

\maketitle

\section*{Introduction}
This document provides a detailed overview of the solutions for the Week 2 Python Programming tasks. Each task is explained with its purpose, the steps taken to develop the solution, the complete Python code, and sample output for specific inputs. The tasks cover fundamental programming concepts including prime numbers, Fibonacci sequences, GCD and LCM calculations, prime factorization, and Kadane's Algorithm for maximum subarray sum.

\section{Task 1: Check if a Number is Prime and List Primes Up to It}
\subsection{Description}
This task requires a program to check if a given number is prime and list all prime numbers up to that number. A prime number is a natural number greater than 1 with no positive divisors other than 1 and itself.

\subsection{Steps to Create}
\begin{enumerate}
    \item Defined a function \texttt{is\_prime(num)} to check if a number is prime by testing divisibility up to its square root.
    \item Created a function \texttt{list\_primes\_up\_to\_n(n)} to iterate from 2 to \texttt{n} and collect prime numbers using \texttt{is\_prime}.
    \item Added input validation to handle non-positive and non-integer inputs.
    \item Used \texttt{math.sqrt} for efficiency in prime checking.
    \item Implemented a main function to get user input, check if the input is prime, and list all primes up to it.
\end{enumerate}

\subsection{Code}
\lstinputlisting{prime_numbers.py}

\subsection{Sample Output}
\begin{verbatim}
Enter a positive integer: 10
10 is not a prime number.
Prime numbers up to 10: [2, 3, 5, 7]
\end{verbatim}

\section{Task 2: Generate First 30 Fibonacci Numbers}
\subsection{Description}
This task involves generating the first 30 Fibonacci numbers using both iterative and recursive approaches. The Fibonacci sequence starts with 0 and 1, where each subsequent number is the sum of the two preceding ones.

\subsection{Steps to Create}
\begin{enumerate}
    \item Implemented \texttt{fibonacci\_iterative(n)} to build the sequence using a loop, appending each new number based on the sum of the previous two.
    \item Created \texttt{fibonacci\_recursive(n)} using a helper function \texttt{fib\_calc(k)} to compute each Fibonacci number recursively.
    \item Handled edge cases (e.g., \texttt{n <= 0}) by returning an empty list or appropriate subset.
    \item Set \texttt{n = 30} as specified and printed results for both approaches.
    \item Included error handling for unexpected issues.
\end{enumerate}

\subsection{Code}
\lstinputlisting{fibonacci.py}

\subsection{Sample Output}
\begin{verbatim}
First 30 Fibonacci numbers (Iterative): [0, 1, 1, 2, 3, 5, 8, 13, 21, 34, 55, 89, 144, 233, 377, 610, 987, 1597, 2584, 4181, 6765, 10946, 17711, 28657, 46368, 75025, 121393, 196418, 317811, 514229]
First 30 Fibonacci numbers (Recursive): [0, 1, 1, 2, 3, 5, 8, 13, 21, 34, 55, 89, 144, 233, 377, 610, 987, 1597, 2584, 4181, 6765, 10946, 17711, 28657, 46368, 75025, 121393, 196418, 317811, 514229]
\end{verbatim}

\section{Task 3: Calculate GCD and LCM}
\subsection{Description}
This task requires functions to compute the Greatest Common Divisor (GCD) and Least Common Multiple (LCM) of two numbers using the Euclidean algorithm.

\subsection{Steps to Create}
\begin{enumerate}
    \item Implemented \texttt{gcd(a, b)} using an iterative Euclidean algorithm, repeatedly dividing and updating until the remainder is zero.
    \item Created \texttt{lcm(a, b)} using the formula \texttt{(a * b) // gcd(a, b)} to ensure integer results.
    \item Used absolute values to handle negative inputs.
    \item Added checks for zero inputs, returning 0 for LCM if either input is zero.
    \item Included input validation for non-integer inputs.
\end{enumerate}

\subsection{Code}
\lstinputlisting{gcd_lcm.py}

\subsection{Sample Output}
\begin{verbatim}
Enter the first integer: 48
Enter the second integer: 36
GCD of 48 and 36: 12
LCM of 48 and 36: 144
\end{verbatim}

\section{Task 4: Find Prime Factors}
\subsection{Description}
This task involves finding all prime factors of a given number, including duplicates (e.g., prime factors of 12 are 2, 2, 3).

\subsection{Steps to Create}
\begin{enumerate}
    \item Defined \texttt{prime\_factors(n)} to find prime factors by dividing by the smallest prime number repeatedly.
    \item Handled the factor 2 separately to optimize by checking only odd numbers afterward.
    \item Continued dividing until the number is reduced to 1, appending each factor to a list.
    \item If the remaining number is greater than 1, added it as a prime factor.
    \item Added input validation for non-positive and non-integer inputs.
\end{enumerate}

\subsection{Code}
\lstinputlisting{prime_factors.py}

\subsection{Sample Output}
\begin{verbatim}
Enter a positive integer: 100
Prime factors of 100: [2, 2, 5, 5]
\end{verbatim}

\section{Task 5: Maximum Sum Subarray (Kadane's Algorithm)}
\subsection{Description}
This task requires finding the subarray with the maximum sum in a list of numbers using Kadane's Algorithm, returning both the sum and the subarray.

\subsection{Steps to Create}
\begin{enumerate}
    \item Implemented \texttt{kadanes\_algorithm(arr)} to track the maximum sum ending at each index (\texttt{current\_sum}) and the overall maximum sum (\texttt{max\_sum}).
    \item Tracked start and end indices to extract the subarray with the maximum sum.
    \item Updated \texttt{current\_sum} as the maximum of the current element or the sum of the current element and previous \texttt{current\_sum}.
    \item Handled empty lists by returning 0 and an empty list.
    \item Parsed comma-separated input and validated for integer values.
\end{enumerate}

\subsection{Code}
\lstinputlisting{kadanes_algorithm.py}

\subsection{Sample Output}
\begin{verbatim}
Enter a list of numbers (comma-separated): -2, 1, -3, 4, -1, 2, 1, -5, 4
Maximum subarray sum: 6
Subarray with maximum sum: [4, -1, 2, 1]
\end{verbatim}

\end{document}